\newSection{Ziel des Versuchs}

\begin{frame}[t]{Abschnitt 1}
    \begin{itemize}
        \item Ermittlung der elektrischen \textbf{Elementarladung} 
        \item experimenteller Nachweis der \textbf{Quantelung} der elektrischen Ladung
    \end{itemize}
\end{frame}

\newSection{Blöcke}
\begin{frame}{Blöcke}
    \begin{definition}[opt. Name]
        Definitionsblock
    \end{definition}
    \pause
    \vspace{1em}
    \begin{theorem}[opt. Name]
        Satz-Block
    \end{theorem}
\end{frame}

\begin{frame}{Blöcke}
    \begin{proof}
        Beweis-Block
    \end{proof}
    \vspace{1em}
    \begin{lemma}[opt. Name]
        Lemma-Block
    \end{lemma}
\end{frame}

\begin{frame}{Blöcke}
    \begin{corollary}[opt. Name]
        Folgesatz-Block
    \end{corollary}
    \vspace{1em}
    \begin{block}{Titel}
        normaler Block mit eignem Titel
    \end{block}
\end{frame}

\begin{frame}{Blöcke}
    \begin{alertblock}{Achtung}
        Alert-Block
    \end{alertblock}
    \begin{exampleblock}{Blockname}
        Beispiel-Block
    \end{exampleblock}
    \begin{example}[opt. Name]
        Beispiel-Block
    \end{example}
\end{frame}

\newSection{Blöcke ohne Titel}
\begin{frame}{Blöcke ohne Titel}
    \begin{block}{}
        normaler Block ohne Titel
    \end{block}
    \begin{exampleblock}{}
        Beispielblock ohne Titel
    \end{exampleblock}
    \begin{alertblock}{}
        Alert-BLock ohne Titel
    \end{alertblock}
\end{frame}
\newSection{Frame-Alignments}
\begin{frame}[t]{an Frame-Anfang angepasst}
    \begin{block}{Schwerkraft}
        \begin{align*}
            F_g = m_{"Ol}\cdot g = \varrho_{"Ol} \cdot V_K \cdot g = \varrho_{"Ol} \cdot \frac{4}{3}\pi \cdot r^3 \cdot g 
        \end{align*}
    \end{block}
\end{frame}

\begin{frame}[c]{an Frame-Mitte angepasst}
    \begin{block}{Schwerkraft}
        \begin{align*}
            F_g = m_{"Ol}\cdot g = \varrho_{"Ol} \cdot V_K \cdot g = \varrho_{"Ol} \cdot \frac{4}{3}\pi \cdot r^3 \cdot g 
        \end{align*}
    \end{block}
\end{frame}

\begin{frame}[b]{an Frame-Ende angepasst}
    \begin{block}{Schwerkraft}
        \begin{align*}
            F_g = m_{"Ol}\cdot g = \varrho_{"Ol} \cdot V_K \cdot g = \varrho_{"Ol} \cdot \frac{4}{3}\pi \cdot r^3 \cdot g 
        \end{align*}
    \end{block}
\end{frame}

\begin{frame}{Angreifende Kräfte}
    \begin{block}{}
        \begin{tabular}{ll}
            Fallen: & $F_g - F_A - F_{R,f} = 0$ \\
            Steigen: & $F_g - F_A - F_E + F_{R,s} = 0$
        \end{tabular}
    \end{block}

    Die \textbf{\textsc{Stoke}sche Reibungskraft} $\pmb{F_R}$ ist proportional zur Geschwindigkeit des Öltröpfchens \\
    \textrightarrow\space Betrag der Reibungskraft ist im Fallen \textbf{größer} als beim Steigvorgang.

    \vspace{1em}
    Es gilt
    \begin{block}{}
        \begin{align*} 
            F_E &= F_{R,f} + F_{R,s}
        \end{align*}
    \end{block}
\end{frame}

\newSection{Bilder}
\begin{frame}{Bild als figure}
    \centering
    \includegraphics[width=0.8\textwidth]{assets/figures/Dämpfung.png}
\end{frame}

\begin{frame}{Bild als input}
    \begin{figure}[!ht]
\centering
\resizebox{\textwidth}{!}{%
\begin{circuitikz}
\tikzstyle{every node}=[font=\large]
\draw [ fill={rgb,255:red,0; green,0; blue,0} ] (6.25,14.25) rectangle (6.5,11.75);
\draw [->, >=Stealth] (9.5,13) -- (5,13);
\draw  (7.5,13) ellipse (0.25cm and 0.75cm);
\draw  (8.75,13) ellipse (0.25cm and 1.25cm);
\draw [short] (6.5,13.75) -- (9.5,11.5);
\draw [short] (6.5,12.25) -- (9.5,14.5);
\draw [line width=1.1pt, short] (9.5,14.75) -- (13.75,14.75);
\draw [line width=1.1pt, short] (9.5,11.25) -- (13.75,11.25);
\draw [short] (13.75,14.75) -- (17,14.75);
\draw [short] (13.75,11.25) -- (17,11.25);
\draw [short] (13.75,11.5) -- (16.75,11.5);
\draw [short] (13.75,14.5) -- (16.75,14.5);
\draw  (17.75,13.5) rectangle (21,12.25);
\draw [short] (17,14.75) -- (17,13.25);
\draw [short] (17,13.25) -- (17.75,13.25);
\draw [short] (17,11.25) -- (17,12.5);
\draw [short] (17,12.5) -- (17.75,12.5);
\draw [short] (16.75,11.5) -- (16.75,14.5);
\draw  (15.5,13) circle (0.75cm);
\draw [short] (15,13.5) -- (16,12.5);
\draw [short] (15,12.5) -- (16,13.5);
\draw [ fill={rgb,255:red,0; green,0; blue,0} ] (11.5,13.75) circle (0.25cm);
\shade[ball color = black!80,opacity = 1] (11.5,13.75) circle (0.25cm);
\draw [->, >=Stealth] (18.5,12.25) -- (18.5,11.75);
\node [font=\large] at (5,12.5) {PC};
\node [font=\large] at (7.75,11) {Mikroskop mit};
\node [font=\large] at (7.75,10.5) {digitalem Okular};
\node [font=\large] at (11.75,13) {Öltröpfchen};
\node [font=\large] at (11.75,10.75) {Kondensator};
\node [font=\large] at (15.5,10.75) {Lampe};
\node [font=\large] at (19.375,12.875) {Steuereinheit};
\node [font=\large] at (18.5,11.5) {PC};
\end{circuitikz}
}%
\caption{Versuchsaufbau\cite{oaverq4milver}}
\end{figure}
\end{frame}